
\documentclass[xcolor=x11names,compress]{beamer}


\usetheme[nocurve]{SmartSerif}

\usepackage{graphicx}
\DeclareGraphicsExtensions{.pdf,.png,.jpg}
\graphicspath{{./images/}}
\usepackage{color}

\newcommand{\noimage}{\includegraphics{placeholder}}


% ---------------------------------------------------------------------------------------
% Thanks to http://www.guidodiepen.nl/2009/07/creating-latex-beamer-handouts-with-notes/
 % \usepackage{handoutWithNotes}
 % \pgfpagesuselayout{4 on 1 with notes}[a4paper,border shrink=5mm]
 % \pgfpagesuselayout{2 on 1 with notes landscape}[a4paper,border shrink=5mm]
% ---------------------------------------------------------------------------------------


\title{Using Life-Logging to Re-Imagine Representativeness in Corpus Design}
\author{Stephen Wattam}
\institute[2013]{Lancaster University}
\date{\tiny \today}

\begin{document}

\maketitle

\frame{\tableofcontents}




% -----------------------------------------------------------------------------------------
\section{Sampling Language}

\subsection{Corpus Aims}
\frame{\frametitle{Corpus Building}
    \begin{itemize}
        \item General-purpose corpora are `reusable samples'

        \item This generality makes constructing a good corpus difficult:

            \begin{itemize}
                \item What is `balanced' or `representative' for one study will not be for another
                \item Only a very large sample will be sufficient for studying rare features
                \item Study-specific resampling and annotation procedures are unknown at the time of sample design
            \end{itemize}
    \end{itemize}
\note{}
}


\frame{\frametitle{Balance/Representativeness}
    \begin{itemize}


        % Representativeness is, simply:
        \item The extent to which a sample resembles the population

        % For a given question 
        % For a given population (needs defining)
            
        % \item The relative propensity for certain values of a useful to be included in a sample

        % \item As such, they vary in statistical suitability for answering various research questions
        \item Requires:
            \begin{itemize}
                \item \textbf{Balance/Proportionality}---The ability of each sample to relate to its real-world counterpart
                    % As seen in some given dimension
                        
                \item \textbf{Size}---The ability to adequately represent sub-populations fully for subsampling and specific inquiry
                    % i.e. is there enough data on a given subtype of language?
            \end{itemize}

        \item Language is so complex that, to satisfy these for a given research question, requires a very large sampe size

    \end{itemize}
\note{}
}


\section{Sampling Design}
\subsection{Conventional Sampling}
\frame{\frametitle{Sampling Design}
    \begin{itemize}
        \item The design of conventional corpora centres around language as a persistent entity
        \item Various variables are extracted and controlled for during sampling (i.e. genre, author, etc.)
        \item For maximum statistical defensibility, these properties should be external
        \item Some disagreement over these salient variables
            % though this won't be my focus here
    \end{itemize}
\note{}
}

\frame{\frametitle{Conventional Sampling}
    
    \noimage{}

% TODO: show a diagram of "look at language use, find lists, cross-reference proportions, sample."
%       [sampling by proxy variables]
\note{}
}



\frame{\frametitle{Difficulties}
    \begin{itemize}
        \item Many variables of interest don't have associated indexes 
            % (i.e. production vs consumption)
        \item Some values of these variables are difficult to sample for ethical or legal reasons
            % i.e. speech vs writing proportions
        \item There is often disagreement between the categories used by builders and users of a corpus
            % and subsampling may not fix this
        \item Variables are selected according to `proxy variables', due to lack of auxiliary data,
            % and these have interactions with other variables of interest, i.e. p[in index list] correlates with some social/linguistic property
        \item 
    \end{itemize}

    \note{}
}


\subsection{Literature}
\frame{\frametitle{Literature \& Criticism}
    \begin{itemize}
        \item Biber
        \item Varadi
        \item Leech
        \item
    \end{itemize}
\note{}
}

% NOTE: another approach here would be to send out questionnaires in order to assess
%       the social validity of corpus results, i.e. to check the indexes.

\subsection{Demographic Sampling}
\frame{\frametitle{Demographic Sampling}
    \begin{itemize}
        \item Selecting different variables of interest re-orients many practical issues
        % back to basics --- what is the aim of CL?
        \item I've chosen to sample language as a social, transitive, event
            % i.e. for a given group of people, what language do they use?
        \item Sample proportions of language use 'in the wild'
        \item With sufficient sample size, this is equivalent, however, pragmatic sampling issues differ greatly
    \end{itemize}
\note{}
}


\frame{\frametitle{Advantages}
    \begin{itemize}
        \item Social demographics are well documented in other samples
        \item The corpus describes ``language use" and population explicitly
        \item Given the same annotation schemes, subsampling is more powerful
            % more variables, i.e. not only author/title/genre (which may still be there)
        \item No central index is required for many sources of text
        \item Some additional variables are exposed to study
            % production/consumption, time/context etc
        \item Sampling of previously unestablished linguistic categories 
            % ephemera, labels, billboards etc
    \end{itemize}

\note{}
}


\frame{\frametitle{New Difficulties}
    \begin{itemize}
        \item Sampling text from many people is technically demanding, and thus expensive
            % We have to leave the office more
            % technology makes this easier nowadays
        \item Acquiring sufficient data for a given demographic now becomes the challenge
            % difficult even for simple data
        \item Coding efforts are helped less by existing indexes/data sources
        \item Multi-modal source requires significant transcription
            % but this can be helped by using the sample as auxiliary data
    \end{itemize}
\note{}
}




\section{Life-Logging}
\subsection{History}
\frame{\frametitle{Life-Logging}
    \begin{itemize}
        \item Continual verbatim recording/broadcasting of one's life
        \item Originally for entertainment only (JenniCam, Justin.tv)
        \item Heavily reliant on video, portable technologies
        \item Spawned 'verbatim memory' projects SenseCam, DARPA's LifeLog
    \end{itemize}
\note{}
}


\subsection{Application to Corpus Design}

\frame{\frametitle{}
    \begin{itemize}
        \item Simple method: follow someone around and collect all language
        \item Sampling is restricted to what can be done realtime only
        \item Deb Roy
    \end{itemize}
\note{}
}



\section{Method}
% TODO: review sampling document
\subsection{Aims/RQs}
\frame{\frametitle{Aims/Sampling Process}
    \begin{itemize}
        \item 
    \end{itemize}
\note{}
}




\section{Data Collection}
% What to record
% 'later lookup'/notes
% trade-off between seamlessness and comprehensiveness
\subsection{Policy}
\frame{\frametitle{Collection Policy}
    \begin{itemize}
        \item 
    \end{itemize}
\note{}
}

\subsection{Methods}
% Methods (one slide each?)
%  - technical
%   - squid
%   - recording (a/v)
%   - blog
%   - notebook
\frame{\frametitle{Methods}
    \begin{itemize}
        \item 
    \end{itemize}
\note{}
}



\subsection{Transcription}
% How to convert and transcribe data from disparate sources
% information processing, organisation, markup
\frame{\frametitle{Operationalisation}
    \begin{itemize}
        \item 
    \end{itemize}
\note{}
}
\frame{\frametitle{Conversion and Transcription}
    \begin{itemize}
        \item 
    \end{itemize}
\note{}
}


\section{Use} 
% Use as auxiliary data
% Resampling
% The idea of gathering more people's info
% Digital-only corpora for web scraping and rabalancing
% subject-specific corpora
\frame{\frametitle{Direct Use}
    \begin{itemize}
        \item NLP models for 'custom' interaction
        \item Tuning of corpora to specific demographics or people
    \end{itemize}
\note{}
}

\frame{\frametitle{Methods for auxiliary data}
    \begin{itemize}
        \item Personal corpora can be used to resample other sources
        \item Indicates the difference between proportional sampling and existing corpus composition
        \item Basis for questionnaires/less intrusive methods to acquire similar data to guide other studies
    \end{itemize}
\note{}
}

% proof-of-concept e corpus stuff 
% \section{Preliminary Work}
% \subsection{E-Corpus}
% \subsection{Pilot Study}


\section{Findings \& Discussion}
\subsection{Findings}
% Check prelim report
\frame{\frametitle{Findings}
    \begin{itemize}
        \item 
    \end{itemize}
\note{}
}


% \section{Scientific \& Ethical}
\frame{\frametitle{Scientific Value}
    \begin{itemize}
        \item Personal Corpus stuff.
    \end{itemize}
\note{}
}

\subsection{Ethics of Covert Methods}
\frame{\frametitle{Ethics}
    \begin{itemize}
        \item 
    \end{itemize}
\note{}
}


% TODO: summary, auxiliary slides

\subsection{Summary}
\frame{\frametitle{Summary}
    \begin{itemize}
        \item
    \end{itemize}
\note{}
}


% -----------------------------------------------------------------------------------------
% 
% \frame{\frametitle{}
% \begin{columns} 
%     \begin{column}[c]{5cm} 
%     \end{column} 
%     \begin{column}[c]{5cm} 
%     \end{column} 
% \end{columns}
% 
% \note{}
% }
%
%
\frame{\frametitle{}
    \begin{itemize}
        \item
    \end{itemize}
\note{}
}
%
% \frame{\frametitle{}
% \note{}
% }
% -----------------------------------------------------------------------------------------
\end{document}
